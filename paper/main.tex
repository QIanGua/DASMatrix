\documentclass[a4paper,10pt,twocolumn]{article}
\usepackage[utf8]{inputenc}
\usepackage{graphicx}
\usepackage{listings}
\usepackage{xcolor}
\usepackage{amsmath}
\usepackage{hyperref}
\usepackage[margin=2cm]{geometry}
\usepackage{booktabs}

\title{DASMatrix: A High-Performance Python Framework for Distributed Acoustic Sensing Data Analysis}
\author{Qianlong \\ Institute of Geophysics and Geomatics, China University of Geosciences}
\date{\today}

\lstset{
    language=Python,
    basicstyle=\ttfamily\small,
    keywordstyle=\color{blue},
    stringstyle=\color{red},
    commentstyle=\color{green!50!black},
    breaklines=true,
    frame=single,
    showstringspaces=false,
    captionpos=b
}

\begin{document}

\maketitle

\begin{abstract}
Distributed Acoustic Sensing (DAS) technology is revolutionizing geophysics and engineering monitoring by transforming fiber optic cables into dense arrays of seismic sensors. However, the high spatial and temporal resolution of DAS generates massive datasets, often reaching terabytes per day, which poses significant challenges for data processing and analysis. Traditional tools often lack the scalability to handle such volumes efficiently or require proprietary software. We present \textbf{DASMatrix}, an open-source, high-performance Python framework designed specifically for DAS data. DASMatrix leverages \textit{lazy loading} and \textit{out-of-core} computing capabilities through Xarray and Dask to process datasets larger than available memory. It provides a fluent, chainable API for intuitive signal processing and supports high-performance visualization. We demonstrate the framework's architecture and performance through case studies, showing its effectiveness in handling large-scale DAS workflows.
\end{abstract}

\section{Introduction}
Distributed Acoustic Sensing (DAS) has emerged as a transformative sensing technology, enabling high-resolution strain rate measurements over long distances using standard fiber optic cables \cite{parker2014distributed}. By interrogating the fiber with laser pulses and analyzing the backscattered Rayleigh light, DAS systems can act as thousands of synchronized sensors.

Despite its potential, the "Big Data" nature of DAS presents a major bottleneck \cite{lindsey2020fiber}. A single DAS interrogator can sample thousands of channels at kilohertz frequencies, generating terabytes of data daily. This volume overwhelms traditional seismic processing workflows that rely on in-memory loading. Furthermore, existing solutions are often fragmented, relying on ad-hoc scripts or expensive proprietary software.

To address these challenges, we introduce \textbf{DASMatrix}, a comprehensive Python framework. Its key contributions are:
\begin{itemize}
    \item \textbf{Scalability}: Built on the Xarray \cite{hoyer2017xarray} and Dask \cite{rocklin2015dask} ecosystem to support out-of-core processing of massive datasets.
    \item \textbf{Usability}: A fluent \texttt{DASFrame} API that simplifies complex processing pipelines.
    \item \textbf{High Performance}: Optimized algorithms for filtering, spectral analysis, and beamforming.
\end{itemize}

\section{System Architecture}
The architecture of DASMatrix is designed with modularity and performance in mind. It consists of four main layers: Acquisition, Core API, Processing, and Visualization.

\subsection{Data Acquisition Layer}
The acquisition layer provides a unified interface for reading various DAS data formats. It supports standard formats like SEG-Y, MiniSEED, and proprietary raw binary formats (DAT, HDF5).
The \texttt{DASReader} class abstracts the file I/O operations. Crucially, it supports \textit{memory mapping}, allowing users to access metadata and signal slices without loading the entire file into RAM.

\subsection{Core API: DASFrame}
At the heart of the framework is the \texttt{DASFrame} class. It wraps an \texttt{xarray.DataArray} but adds DAS-specific semantics (distance, time, channel coordinates).
\begin{lstlisting}[caption=Creating a DASFrame]
from DASMatrix import df
# Lazy load a large HDF5 file
frame = df.read("large_dataset.h5")
# Frame is lightweight; data is not loaded yet
print(frame)
\end{lstlisting}

\subsection{Processing Engine}
The processing engine leverages Dask to build a computation graph. Operations like filtering, detrending, and integration are added to the graph lazily. Execution is triggered only when \texttt{.collect()} or a visualization method is called. This strategy enables the processing of datasets that exceed the machine's physical memory.

\section{Implementation Details}

\subsection{Lazy Evaluation \& Chaining}
DASMatrix implements a fluent interface where methods return a new \texttt{DASFrame} instance (or modify the internal graph). This allows for clean, readable code:
\begin{lstlisting}[caption=Chainable Processing Pipeline]
processed = (
    frame
    .detrend(axis="time")
    .bandpass(low=1.0, high=50.0)
    .normalize()
)
\end{lstlisting}
Under the hood, \texttt{.bandpass} applies a Dask \texttt{map\_overlap} operation to handle filtering across chunk boundaries seamlessly without edge artifacts.

\subsection{High-Performance Kernels}
For computationally intensive tasks, we utilize vectorized NumPy \cite{harris2020array} operations and, where necessary, JIT compilation via Numba \cite{lam2015numba} to approach C-level speeds in pure Python.

\section{Case Studies}

\subsection{Spectral Analysis Workflow}
A common task in DAS is analyzing the frequency content of traffic or seismic signals. DASMatrix simplifies this to a few lines:
\begin{lstlisting}
# Compute STFT spectrogram
frame.stft(nperseg=256).plot(cmap='inferno')
\end{lstlisting}
The framework automatically handles windowing, overlaps, and plotting axes.

\subsection{F-K Filtering}
To separate wavefields based on velocity, Frequency-Wavenumber (F-K) filtering is essential. DASMatrix provides an intuitive interface for this 2D transform operation.
\begin{lstlisting}[caption=F-K Velocity Filtering]
# Remove low-velocity noise (v < 1000 m/s)
# dx specifies channel spacing in meters
filtered = frame.fk_filter(v_min=1000.0, dx=10.0)
\end{lstlisting}
The framework handles the forward F-K transform, mask generation based on the velocity thresholds, and the inverse transform transparently.

\subsection{Complete Processing Pipeline}
A typical real-world workflow involves chaining multiple operations. In this example, we process a raw DAS file to enhance traffic signals:

\begin{lstlisting}[caption=End-to-End Processing Pipeline]
from DASMatrix import df

# 1. Lazy load raw data
frame = df.read("traffic_data.h5")

# 2. Build processing graph
processed = (
    frame
    .detrend()                 # Remove DC offset
    .bandpass(5.0, 100.0)      # Isolate traffic band
    .fk_filter(v_min=300)      # Remove air-waves
    .normalize()               # AGC
)

# 3. Compute and Visualize
# Only now is data loaded and processed
processed.plot_waterfall(
    title="Enhanced Traffic Flow",
    cmap="seismic"
)
\end{lstlisting}
This pipeline demonstrates the power of the fluent API, allowing researchers to express complex signal processing logic in a clear, declarative manner.

This pipeline demonstrates the power of the fluent API, allowing researchers to express complex signal processing logic in a clear, declarative manner.

\section{Performance Evaluation}
To validate the performance claims, we conducted a benchmark comparing DASMatrix's processing engine against a standard in-memory NumPy/SciPy implementation.

\subsection{Benchmark Setup}
The benchmark simulated a typical preprocessing workflow consisting of:
\begin{enumerate}
    \item Linear Detrending (Time axis)
    \item Absolute Value Calculation
    \item Amplitude Scaling
\end{enumerate}
The test dataset consisted of 50,000 time samples $\times$ 2,000 channels (approx. 400 MB float32 data). The test was performed on a standard workstation CPU.

\subsection{Results}
The results demonstrate a significant speedup using DASMatrix's graph-based execution engine compared to the eager execution of NumPy:

\begin{table}[h]
\centering
\caption{Processing Time Comparison (lower is better)}
\begin{tabular}{lcc}
\toprule
\textbf{Implementation} & \textbf{Time (s)} & \textbf{Speedup} \\
\midrule
Standard NumPy/SciPy & 10.89 & 1.00x \\
DASMatrix (Warm Start) & \textbf{5.74} & \textbf{1.90x} \\
\bottomrule
\end{tabular}
\end{table}

The \textbf{1.90x} speedup is attributed to operation fusion (reducing memory access overhead) and optimized Dask chunking. Furthermore, while the NumPy implementation requires loading the entire dataset into RAM, the DASMatrix implementation can scale to datasets terabytes in size without memory exhaustion, thanks to its out-of-core design.

\section{Conclusion}
DASMatrix fills a critical gap in the DAS research community by providing a free, open-source, and high-performance tool. By leveraging modern Python data science stack components like Xarray and Dask, it democratizes access to large-scale DAS data analysis, enabling researchers to focus on geophysics rather than data engineering. Future work will focus on integrating deep learning workflows directly into the pipeline.

\bibliographystyle{plain}
\bibliography{references}

\end{document}
